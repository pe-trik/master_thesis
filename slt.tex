\chapter{Spoken Language Translation}
\label{chap:slt}
In the preceding chapter, we introduced and trained an enhanced ASR. The pipeline of the enhanced ASR system consists of an acoustic model and a P2G translation model. In the case of the ASR, the translation model rewrites phonetic transcriptions to graphemes in the language of the input. 

In this chapter, we develop a spoken language translation pipeline for Czech and English language --- in both directions. We build the pipeline architecture upon previously proposed Enhanced ASR. From the Enhanced ASR, we take the acoustic model, which remains identical. The essential difference is the change of target language for the translation model. 

We organize this chapter as follows: first, we analyze relevant work on SLT in \cref{slt:related}. Next follows our motivation and experiment outline in \cref{slt:outline}. In \cref{slt:training} we describe the training process and in \cref{slt:evaluation} we evaluate the trained systems. Finally, we conclude the experiment in \cref{slt:conclusion}.

\section{Related Work}
\label{slt:related}

With the advent of deep learning, we observe a clear trend towards end-to-end models in SLT.

For instance, \perscite{weiss2017sequence} presents an attention-based model. They use the very architecture of ASR --- encoder, and decoder. Their end-to-end trained network outperforms ASR-MT cascade on Spanish-English speech translation task.

Our great inspiration in our work and particularly for our spoken language translation, is the study of \perscite{salesky-etal-2019-exploring}. They examine an alternative speech feature representation for the end-to-end speech translation. The authors propose to use compressed phoneme-like speech representation. Their method first generates phoneme labels for a speech utterance. Second, consecutive frames with the same label are averaged-out. To compute the phoneme labels, authors use a hybrid HMM-DNN system implemented in Kaldi \parcite{Kaldi}. For the translation, they use a sequence-to-sequence model based on LSTM. The authors observe an improvement of the translation quality up to 5 BLEU on low- and high-resource language pairs. The main difference of our work from Salesky is that we use NN ASR and Transformer translation step.


\section{Motivation and Experiment Outline}
\label{slt:outline}
We propose to split the SLT pipeline to the acoustic and SLT model (the architecture is identical with \cref{fig:asr_enhanced_pipeline}). As the intermediate representation, we choose phonemes. We are highly motivated by their success in the \perscite{salesky-etal-2019-exploring}. Also, as reviewed in earlier chapters on ASR (\cref{chapter:asr,chap:enhanced_asr}), phonemes are competitive representation units in speech processing.  

In our spoken language translation pipeline, we take the acoustic model from \cref{chapter:asr}. Specifically, we use the ASR to phonemes. As a translation model, we utilize the Transformer architecture.

We compare our proposed SLT system to a more conventional SLT pipeline --- an ASR-MT tandem.

\section{Training}
\label{slt:training}
Following the experiment outline as described in the preceding section, we train two translation models (Czech phonemes to English graphemes and English phonemes to Czech graphemes). Additionally, we also train baseline grapheme-to-grapheme translation models (English and Czech, in both directions). 

The training procedure remains almost identical to the training scheme in the Enhanced ASR (see \cref{easr:training}). As we concluded in the \cref{easr:conclusion}, the bigger Transformer is a better fit for the ASR task. Furthermore, the translation task to a different language is an even more difficult problem. Accordingly, we work only with the Transformer ``big'' in this chapter.

For the evaluation during the training of phoneme-to-grapheme models we use \texttt{news\-test\-2015}. We phonemize the source side of the data set using \texttt{phonemizer}.

Again, as determined in \cref{easr:tok_conclusion}, we use separate vocabularies for source and target languages. Correspondingly, we use 32k vocabulary size. Source BPE is trained on a clean, phonemized CzEng dataset. The target is trained on clean original CzEng.

For training, we utilize the following hyper-parameters for Transformer ``big'':\\

\begin{tabular}{ll}
	 GPU(s): & 10 GPUs with at least 11 GB VRAM \\
	 batch size:& 2000\\
	 learning rate:& 0.03\\
	 warm-up steps:& 8000\\
	 max steps:& 600000 with manual abortion\\
\end{tabular}\\

To attain better regularization, we involve the BPE drop-out technique proposed by \perscite{provilkov2019bpe}. We set the drop-out probability to 0.1, following the best setting from the original paper. The BPE drop-out is applied to both, the source and target of the translation.

The progression of training is in \cref{fig:cs_en} for Czech to English and in \cref{fig:en_cs} for English to Czech as evaluations on \texttt{newstest2015}.

\begin{figure}[h]
	\includegraphics[width=\linewidth,height=7cm]{img/cs_en}
	\caption[Czech to English training]{Evaluations on newstest2015 during the training. Phonemized Czech side as the source and (original, in graphemes) English side as the target.}
	\label{fig:cs_en}
\end{figure}

\begin{figure}[h]
	\includegraphics[width=\linewidth,height=7cm]{img/en_cs}
	\caption[English to Czech training]{Evaluations on newstest2015 during the training. Phonemized English side as the source and (original, in graphemes) Czech side as the target.}
	\label{fig:en_cs}
\end{figure}

\section{Evaluation}
\label{slt:evaluation}
In this section, we evaluate the trained models as part of the proposed spoken language translation pipeline (the pipeline scheme is identical to that of the enhanced ASR, see the scheme in \cref{fig:asr_enhanced_pipeline} on page \pageref{fig:asr_enhanced_pipeline}).

We evaluate the SLT pipeline consisting of an ASR and an NMT model. More precisely, our proposed pipeline includes an acoustic model (i.e. a model transcribing speech to phonemes) and a translation model (i.e., an NMT from phonemes in source language to graphemes in the target language). 

Note that all models used in this comparison (baseline and proposed) are trained on the same datasets. The only contrast is the phonemization of transcripts for the proposed acoustic model and translation source for the proposed SLT translation model.

Again, as in the previous chapter, we apply different beam sizes during the evaluation.

\subsection{SLT Test Set Preparation}
\label{read-newstest}
To the best of our knowledge, there is no publicly available SLT test set from Czech to English. We therefore decided to create one.

We base our test set on a subset of the \texttt{newstest2018}. The original \texttt{news\-test\-2018} contains more than 3000 sentences. The test set is a collection of article passages from various news. There are articles of Czech origin, but also of English one. For our purposes, we selected only the articles of the Czech origin. 

A re-speaker was instructed to use our online tool \texttt{Voice Recorder}.\footnote{We developed this tool within the ELITR project (\url{elitr.eu}). Source code of the recorder is available at \url{https://github.com/ELITR/voice-recorder}} The interface displays one sentence at time. The user can read the sentence, record it and replay. If she or he is not satisfied with the recording, they may record a new version. With this workflow, the resulting audio is produced as already segmented into individual sentences.

We employed one re-speaker --- a Czech woman. She recorded 747 sentences in English and 747 sentences in Czech. This resulted in 1:18:35 English, and 1:11:38 of Czech read speech audio.

We evaluated the performance of our acoustic models on obtained test set. The results are in the \cref{tab:acoustic_eval}. 

We were quite surprised by the poor performance of Czech acoustic models --- 34.33 \% Phoneme Word Error Rate (PWER) and 34.51 \% Word Error Rate (WER). We believe this is due to the nature of the training data we used for the training of the models. The Corpus of Czech Parliament Plenary Hearings has a unique acoustic. Further, there is a high prevalence of man speakers. And finally, the size of the training corpus is rather small --- only 400 hours.

The English models performed better than the Czech ones --- 18.61 \% PWER and 18.60 \% WER. First, the models are trained on approximately 2000 hours of speech. Also, the Common Voice dataset that is used for traing of the models contains many non-native speakers.

Still, the performance of the models is rather inferior to the test sets in well-know datasets (such as LibriSpeech or Common Voice). We believe, this is partially due to the high frequency of Czech proper nouns and specificity of the texts origin (news articles). 

\begin{table}[t]
	\centering
	\begin{tabular}{lc|c}
		& Model alphabet & Performance   \\ \hline
		\multirow{2}{*}{Czech}   & phonemes       & 34.33 \% PWER \\
		& graphemes      & 34.51 \% WER  \\ \hline
		\multirow{2}{*}{English} & phonemes       & 18.61 \% PWER \\
		& graphemes      & 18.60 \% WER 
	\end{tabular}
	\caption[Acoustic models performance on \texttt{read-newstest}]{Evaluation of acoustic models on our test set derived from \texttt{news\-test\-2018}.}
	\label{tab:acoustic_eval}
\end{table}

\subsection{Beam Size} 
SacreBLEU scores in the \cref{tab:eval_slt,tab:eval_slt} are consistent with the ones commonly used in the neural machine translation. We see that the models are, in general, confident at each step and do not need to consider other alternatives. Enlarging the beam size beyond a small number leads to performance deterioration. Having a beam size of one (effectively a greedy decoding) also leads to sub-optimal translations, so the optimum beam size in our experiments turned out to be 4.

The only exception to this is the English to Czech proposed SLT. The beam size of 16 outperformed other beam sizes. However, the margin is relatively small --- 0.23 BLEU. The model is probably less confident when the source is corrupted as a result of processing a real sound with the acoustic model.

\subsection{Czech to English}
For evaluation of the Czech to English SLT pipeline, we take the acoustic model from \cref{asr:transfer_phonemes}. This model has Phoneme Word Error Rate 8.94 \% on the phonemized Corpus of Czech Parliament Plenary Hearings.

We compare our proposal with a more typical setup --- a standard ASR followed by an NMT. We employ the best performing ASR model from \cref{asr:crosslingual_intermediate}. This model has a word error rate of 7.64 \% on the Corpus of Czech Parliament Plenary Hearings.

\cref{tab:eval_slt} contains SacreBLEU scores for Czech to English SLT. Translation samples are in \cref{tab:cs_en_names,tab:cs_en_sample}.

Additionally to the SLT pipeline evaluation, we also evaluate performance of the models on P2G translation task alone. We take the true transcripts (phoneme transcripts for the proposed model and grapheme transcripts for the baseline model) and use them as the source of the SLT translation systems.

First, we evaluate the SLT task. We observe that the proposed phoneme-based system outperforms the baseline. The margin is rather small in terms of the SacreBLEU metric.

In the translation task (with correct transcripts as sources), the baseline closely outperforms the proposed system. This is probably due to the easier detection and consecutive translation of proper nouns in graphemes.

We see that the errors introduced by the acoustic/ASR model considerably demage the translation quality --- 11 points of SacreBLEU. A probable cause of such a drastic SacreBLEU score reduction is the high (phoneme) word error rate of the acoustic and ASR models (more than 30 \%). This is quite surprising, as the re-speaker is a Czech native.

\paragraph{Manual Evaluation}
We also manually assess the translation quality obtained in the SLT evaluation.

We design the evaluation procedure as a blind experiment. We use \texttt{QuickJudge}\footnote{\url{https://github.com/ufal/quickjudge}} and proceed as follows:

\begin{itemize}
	\item we evaluate only a random sub-sample of the whole test set,
	\item we shuffle the order of segments (tuples consisting of the source, ground truth and hypotheses) in the test set,
	\item order of hypotheses in a segment is randomly switched (to prevent favoring one particular system across the test set),
	\item we evaluate the first 50 samples, 
	\item we annotate hypotheses as follows:
	\begin{itemize}
		\item ``\texttt{**}'' for correct translation,
		\item ``\texttt{*}'' for almost correct translation,
		\item nothing for incorrect translation,
	\end{itemize}
	\item we consider these criteria (in the given order):
	\begin{enumerate}
		\item content,
		\item validity, or at least, phonological similarity of proper nouns,
		\item grammar.
	\end{enumerate}
\end{itemize}

\begin{landscape}
	\begin{table}[]
		\centering
		\begin{tabular}{lc|cccc|ccc|ccc}
			&
			\multicolumn{1}{l|}{} &
			\multirow{3}{*}{\textbf{Model}} &
			\multirow{3}{*}{\textbf{Source}} &
			\multirow{3}{*}{\textbf{Source error}} &
			\multirow{3}{*}{\textbf{Target}} &
			\multicolumn{3}{c|}{\textbf{Cased, interpuct.}} &
			\multicolumn{3}{c}{\textbf{Uncased, no interpunct.}} \\

			&\multicolumn{1}{l|}{}  &       &       &           &          & \multicolumn{3}{c|}{Beam Size} & \multicolumn{3}{c}{Beam Size} \\

			&\multicolumn{1}{l|}{}  &       &       &           &          & 1        & 4        & 16       & 1        & 4        & 16       \\ \cline{2-12}

			\multirow{4}{*}{\textbf{Czech $\rightarrow$ English}}&\multicolumn{1}{c|}{\multirow{2}{*}{\begin{tabular}[c]{@{}c@{}}\textbf{SLT}\\ (ASR source)\end{tabular}}} &
			P2G &
			ASR Phon &
			PWER 34.33 \% &
			\multirow{2}{*}{En Graph} & 19.47 & \textbf{20.07} & 19.75 & 19.56 & \textbf{19.96} & 19.43 \tabspace{16pt}\\

			&\multicolumn{1}{c|}{} &  baseline & ASR Graph & WER 34.51 \% &   & 18.39 & 19.69 & 19.01 & 18.5 & 19.74 & 18.99    \\[0.2\normalbaselineskip] \cline{2-12}

			&\multicolumn{1}{c|}{\multirow{2}{*}{\begin{tabular}[c]{@{}c@{}}\textbf{Translation}\\ (clean source)\end{tabular}}} &
			P2G &
			Clean Phon &
			\multirow{2}{*}{0 \%} &
			\multirow{2}{*}{En Graph} & 30.14 & 30.82 & 30.35 & 30.44 & 30.82 & 30.56 \tabspace{16pt}\\
			
			&\multicolumn{1}{c|}{} &  baseline & Clean Graph & & & 30.31 & \textbf{31.07} & 30.45 & 30.74 & \textbf{31.55} & 30.77 \\

			\midrule

			\multirow{4}{*}{\textbf{English $\rightarrow$ Czech}} & \multicolumn{1}{c|}{\multirow{2}{*}{\begin{tabular}[c]{@{}c@{}}\textbf{SLT}\\ (ASR source)\end{tabular}}} &
			P2G &
			ASR Phon &
			PWER 18.61 \% & \multirow{2}{*}{Cs Graph} & 16.64 & 17.12 & \textbf{17.35} & 14.28 & 14.76 & \textbf{14.9}\tabspace{16pt}\\
			
			&\multicolumn{1}{c|}{} &  baseline & ASR Graph & WER 18.80 \% &   & 16.51 & 16.83 & 16.68 & 14.29 & 14.53 & 14.42    \\[0.2\normalbaselineskip] \cline{2-12}

			&\multicolumn{1}{c|}{\multirow{2}{*}{\begin{tabular}[c]{@{}c@{}}\textbf{Translation}\\ (clean source)\end{tabular}}} &
			P2G &
			Clean Phon &
			\multirow{2}{*}{0 \%} &
			\multirow{2}{*}{Cs Graph} & 21.75 & 22.32 & 22.26 & 19.6 & 20.01 & 20.06 \tabspace{16pt}\\
			
			&\multicolumn{1}{c|}{} &  baseline & Clean Graph & & & 21.85 & \textbf{22.47} & 22.21 & 19.74 & \textbf{20.09} & 20.07
		\end{tabular}
		\caption[SLT evaluation]{Evaluation of the proposed models (phonemes to graphemes --- P2G) and the baseline ones (graphemes to graphemes). We evaluate performance on SLT and Translation task. SLT task obtained source from ASR transcripts. Translation task is done on clean (original) source.}
		\label{tab:eval_slt}
	\end{table}
\end{landscape}

\iffalse
\begin{landscape}
	\begin{table}[]
		\centering
		\begin{tabular}{c|cccc|ccc|ccc}
			\multicolumn{1}{l|}{} &
			\multirow{3}{*}{\textbf{Model}} &
			\multirow{3}{*}{\textbf{Source}} &
			\multirow{3}{*}{\textbf{Source error}} &
			\multirow{3}{*}{\textbf{Target}} &
			\multicolumn{3}{c|}{\textbf{Cased, interpuct.}} &
			\multicolumn{3}{c}{\textbf{Uncased, no interpunct.}} \\
			\multicolumn{1}{l|}{}  &       &       &           &          & \multicolumn{3}{c|}{Beam Size} & \multicolumn{3}{c}{Beam Size} \\
			\multicolumn{1}{l|}{}  &       &       &           &          & 1        & 4        & 16       & 1        & 4        & 16       \\ \hline
			\multicolumn{1}{c|}{\multirow{2}{*}{\begin{tabular}[c]{@{}c@{}}\textbf{SLT}\\ (ASR source)\end{tabular}}} &
			P2G &
			ASR Phon &
			PWER 34.33 \% &
			\multirow{2}{*}{En Graph} & 19.47 & \textbf{20.07} & 19.75 & 19.56 & \textbf{19.96} & 19.43 \tabspace{16pt}\\
			\multicolumn{1}{c|}{} &  baseline & ASR Graph & WER 34.51 \% &   & 18.39 & 19.69 & 19.01 & 18.5 & 19.74 & 18.99    \\[0.2\normalbaselineskip] \hline
			\multicolumn{1}{c|}{\multirow{2}{*}{\begin{tabular}[c]{@{}c@{}}\textbf{Translation}\\ (clean source)\end{tabular}}} &
			P2G &
			Clean Phon &
			\multirow{2}{*}{0 \%} &
			\multirow{2}{*}{En Graph} & 30.14 & 30.82 & 30.35 & 30.44 & 30.82 & 30.56 \tabspace{16pt}\\
			
			\multicolumn{1}{c|}{} &  baseline & Clean$\dagger$ Graph & & & 30.31 & \textbf{31.07} & 30.45 & 30.74 & \textbf{31.55} & 30.77
		\end{tabular}
		\caption[Czech to English SLT evaluation]{Evaluation of the proposed Czech to English model (phonemes to graphemes --- P2G) and the Czech to English baseline (graphemes to graphemes). We evaluate performance on SLT and Translation task. SLT task obtained source from ASR transcripts. Translation task is done on clean (original) source.\\$\dagger$ ASR-like Graph is original lowercase source with stripped punctuation.}
		\label{tab:eval_slt}
	\end{table}
\end{landscape}
\fi

\begin{table}[H]
	\centering
	\small
	\begin{tabular}{l|llll}
		\textbf{Original} & Tomáš Rosický & František Savov  & Karim Rashid     & Karel Štědrý       \\ \hline
		\textbf{Proposed} & Thomas Rozicki     & Francis Sava & Cary Rashid      & Karl the Generous  \\
		\textbf{Baseline} & Tomas Rizky    & Frantise Saho    & Cary Rash & Kanel the Generous
	\end{tabular}
	\caption{Samples of names from Czech to English SLT.}
	\label{tab:cs_en_names}
\end{table}

\begin{table}[H]
	\centering
	\begin{tabular}{ll}
		\multicolumn{1}{l|}{\textbf{Source}}   & Obviněný podle všeho nad svým činem lítost neprojevoval.      \\
		\multicolumn{1}{l|}{\textbf{Target}}   & The accused apparently showed no regret for his actions. \\
		\multicolumn{1}{l|}{\textbf{Proposed}} & The accused seemed to show little remorse for his or her lack of pity.      \\
		\multicolumn{1}{l|}{\textbf{Baseline}} & The accusation seemed to show no regret for his actions.    \\
		\textbf{}                              &                                                           \\
		\multicolumn{1}{l|}{\textbf{Source}}   & \makecell[l]{Zakázku poprvé vysoutěžila v roce 2013, kdy ministerstvo průmyslu\\ a obchodu vedl Martim Kuba ODS.}      \\
		\multicolumn{1}{l|}{\textbf{Target}}   & \makecell[l]{The competition was first launched in 2013 when the Department of\\ Industry and Trade was led by Martim Kuba of the ODS party.}      \\
		\multicolumn{1}{l|}{\textbf{Proposed}} & \makecell[l]{The contract was first won in 2013 when Martin Cuba OTS was in\\ charge of the Ministry of Industry and Trade.}                 \\
		\multicolumn{1}{l|}{\textbf{Baseline}} & \makecell[l]{She first competed for the contract in 2,13, when the Department of\\ Industry and Trade led Martin Cuba from...}      \\
		\textbf{}                              &                                                           \\
		\multicolumn{1}{l|}{\textbf{Source}}   & \makecell[l]{Dvojice mladých Rusů Ilja Žirnov a Kira Čerkasovová byla\\ pohřešována od prosince 2005.}      \\
		\multicolumn{1}{l|}{\textbf{Target}}   & \makecell[l]{The two young Russians, Ilja Zhirnov and Kira Cherkasov, had been\\ missing since December 2005.} \\
		\multicolumn{1}{l|}{\textbf{Proposed}} & \makecell[l]{A pair of small Russians, Ilya Zirnov Akira Cherkasova, were missing\\ since December 2005.}      \\
		\multicolumn{1}{l|}{\textbf{Baseline}} &\makecell[l]{Ilya Zernov, Akiracergasov, has been missing since December.}   
	\end{tabular}
	\caption[Output samples from Czech to English SLT]{Output samples from Czech to English SLT. Note the typo Marti\textit{m} in the source and reference of the second sentence. This common Czech name was correctly recognized by both shown systems. Additionally, the target is incorrectly translated and should be: ``It (presumably a company) won the tender for the first time in 2013... ''.}
	\label{tab:cs_en_sample}
\end{table}



\begin{table}[t]
	\centering
	\begin{tabular}{l|ccc}
		& Correct & Almost correct & Incorrect \\ \hline
		Proposed      & 12      & 14             & 24        \\
		Baseline      & 7       & 15             & 28        \\ \hline
		Only$\dagger$ proposed & 7       & 4              & 7         \\
		Only$\dagger$ baseline & 2       & 5              & 11        \\ \hline
		Both          & 1       & 6              & 17       
	\end{tabular}	
	\caption[Czech to English manual evaluation]{Results of ratings of manual annotation ofn the Czech to English SLT. We reviewed 50 randomly selected samples.\\$\dagger$ Only means, that in case of (almost) correct translations the other system produced incorrect translation and vice versa.}
	\label{tab:manual_cs_en}
\end{table}

The distribution of manual annotations of the Czech to English SLT is in \cref{tab:manual_cs_en}. 

As suggested by the SacreBLEU score, the proposed system outperforms the baseline in manual annotation. The proposed SLT system produces a correct translation in more cases than the baseline (12 correct, 14 almost correct compared to 7 correct and 15 almost correct). However, we must acknowledge that the overall performance of both systems is rather poor --- in 24 cases (out of 50) the proposed system and in the 28 cases the baseline one failed to provide acceptable translations.

We discovered a few notable differences while reviewing the translations. First, we find a somewhat better handling of proper nouns by the proposed system as very important. The translation are rarely without errors, but they are correctly recognized as proper nouns. Further, they tend to have a pronunciation closer to the original than those of the baseline system.

Also, if the system fails to recognize a word, it is commonly rendered as a proper noun (e.g., ``Herec byl tehdy o 12 let starší...'', ``The actor was 12 years older then...'' translates as ``Hannes was twelve years older...''). This is a bit distracting, but we see a possibility that in a broader context, the word can be understood by the user.

Slightlz better recognition of numbers is also interesting. This is probably due to a side-effect of the phonemization. The CzEng corpus is not consistent in using numbers. The \texttt{phonemizer} rewrites all numbers written using numeric characters to words.

\subsection{English to Czech}
In order to evaluate the English to Czech translation model as part of the proposed SLT pipeline, we take the English acoustic model from \cref{asr:transfer_phonemes}. The model has a phoneme word error rate 4.18 \% on LibriSpeech \texttt{test-clean} and 11.48 \% on the \texttt{test-other}.

The baseline SLT pipeline uses as ASR model the Jasper model trained on LibriSpeech and Common Voice. We downloaded the model from the NVIDIA NGC.\footnote{\url{https://ngc.nvidia.com/catalog/models/nvidia:multidata\%20set\_jasper10x5dr}}

We also assess the translation models performance on the P2G translation task (i.e., the translation sources are true transcripts, not burdened with errors introduced by the acoustic/ASR models).

\cref{tab:eval_slt} contains SacreBLEU scores of the English to Czech translation models as part of a SLT pipeline and in the translation task. Translation samples are in \cref{tab:en_cs_names,tab:en_cs_sample}.

Similarly, as the Czech to English SLT, the proposed system (with phonemes as an intermediate step) outperforms the baseline. Again, the difference is rather small --- 0.52 BLEU. 

\begin{table}[t]
	\centering
	\begin{tabular}{l|ccc}
		& Correct & Almost correct & Incorrect \\ \hline
		Proposed      & 16      & 21             & 13        \\
		Baseline      & 18       & 18             & 14        \\ \hline
		Only$\dagger$ proposed & 6       & 2              & 7         \\
		Only$\dagger$ baseline & 6       & 1              & 8        \\ \hline
		Both          & 4       & 11              & 6       
	\end{tabular}	
	\caption[English to Czech manual evaluation]{Results of manual annotation of the English to Czech SLT. We reviewed 50 randomly selected samples.\\$\dagger$ Only means, that in case of (almost) correct translations the other system produced incorrect translation and vice versa.}
	\label{tab:manual_en_cs}
\end{table}


We manually review the quality of the both SLT systems. The distribution of manual annotations of the Czech to English SLT is in the \cref{tab:manual_en_cs}. 

Compared to the opposite direction (Czech to English), the overall subjective translation quality improved significantly. We found only 13 and 14 translations respectively incorrect. Although the direction to Czech language is more challenging in general \XXX{do I need to support this?}, we explain this reduction of incorrect translations as the result of a better acoustic and ASR model.

In the random sub-sample that we evaluated, we found the translations of the baseline to be slightly better. Still, the difference is relatively small. The baseline produced more ``correct'' translations, while the proposed produced less ``incorrect'' (``correct'' + ``almost correct'') translations. 

We discovered that the baseline tends to mark unknown words as proper nous (e.g.: ``This species lives...'' translates as ``Tento Pecies žije...'', or ``...the murderer's parents...'' as ``rodičů Murdereora''). We marked such translation as ``almost correct'', as we think that such words can be understood in the context. Similarly, the proposed system does so, but more frequently around proper nous (e.g., ``...in the new Prima series Kapitán Exner'' as ``...v nové Primma Series Capitan Expert'').

With the overall translation quality, the handling adequacy of proper nouns also rises (see the \cref{tab:en_cs_names}). It is worth noting that the re-speaker was a native Czech. The Czech proper nouns are hence correctly pronounced in Czech. Therefore, we find it quite interesting, that both system were able to translate the name satisfactorily.

\iffalse
\begin{landscape}
	\begin{table}[]
		\centering
		\begin{tabular}{c|cccc|ccc|ccc}
			\multicolumn{1}{l|}{} &
			\multirow{3}{*}{\textbf{Model}} &
			\multirow{3}{*}{\textbf{Souce}} &
			\multirow{3}{*}{\textbf{Souce error}} &
			\multirow{3}{*}{\textbf{Target}} &
			\multicolumn{3}{c|}{\textbf{Cased, interpuct.}} &
			\multicolumn{3}{c}{\textbf{Uncased, no interpunct.}} \\
			\multicolumn{1}{l|}{}  &       &       &           &          & \multicolumn{3}{c|}{Beam Size} & \multicolumn{3}{c}{Beam Size} \\
			\multicolumn{1}{l|}{}  &       &       &           &          & 1        & 4        & 16       & 1        & 4        & 16       \\ \hline
			\multicolumn{1}{c|}{\multirow{2}{*}{\begin{tabular}[c]{@{}c@{}}\textbf{SLT}\\ (ASR source)\end{tabular}}} &
			P2G &
			ASR Phon &
			PWER 18.61 \% &
			\multirow{2}{*}{Cs Graph} & 16.64 & 17.12 & \textbf{17.35} & 14.28 & 14.76 & \textbf{14.9}\tabspace{16pt}\\
			\multicolumn{1}{c|}{} &  baseline & ASR Graph & WER 18.80 \% &   & 16.51 & 16.83 & 16.68 & 14.29 & 14.53 & 14.42    \\[0.2\normalbaselineskip] \hline
			\multicolumn{1}{c|}{\multirow{2}{*}{\begin{tabular}[c]{@{}c@{}}\textbf{Translation}\\ (clean source)\end{tabular}}} &
			P2G &
			Clean Phon &
			\multirow{2}{*}{0 \%} &
			\multirow{2}{*}{Cs Graph} & 21.75 & 22.32 & 22.26 & 19.6 & 20.01 & 20.06 \tabspace{16pt}\\
			
			\multicolumn{1}{c|}{} &  baseline & Clean$\dagger$ Graph & & & 21.85 & \textbf{22.47} & 22.21 & 19.74 & \textbf{20.09} & 20.07
		\end{tabular}
		\caption[English to Czech SLT evaluation]{Evaluation of the proposed English to Czech model (phonemes to graphemes --- P2G) and the English to Czech baseline (graphemes to graphemes). We evaluate performance on SLT and Translation task. SLT task obtained source from ASR transcripts. Translation task is done on clean (original) source.\\$\dagger$ ASR-like Graph is original lowercase source with stripped punctuation.}
		\label{tab:eval_slt_en_cs}
	\end{table}
\end{landscape}
\fi

\begin{table}[]
	\centering
	\small
	\begin{tabular}{l|llll}
		\textbf{Original} & Mádlová & Donald Trump  & Lucie Bílá     & Simona Krainová       \\ \hline
		\textbf{Proposed} & Madlova     & Donald Tramp & Rusi-Billa      & Simana Cranova  \\
		\textbf{Baseline} & Madlovová    &  Donald Trumf   & Rutiabilia & Simona Cranova
	\end{tabular}
	\caption{Samples of names from English to Czech SLT.}
	\label{tab:en_cs_names}
\end{table}

\begin{table}[]
	\centering
	\begin{tabular}{ll}
		\multicolumn{1}{l|}{\textbf{Source}}   & Now it is a finished product that is produced in our country.      \\
		\multicolumn{1}{l|}{\textbf{Target}}   & Nyní jde o hotový produkt, který je navíc vyráběný u nás. \\
		\multicolumn{1}{l|}{\textbf{Proposed}} & Nyní je to hotový produkt, který se vyrábí v naší zemi.      \\
		\multicolumn{1}{l|}{\textbf{Baseline}} & Nyní se jedná o hotový produkt vyráběný v naší zemi.    \\
		\textbf{}                              &                                                           \\
		\multicolumn{1}{l|}{\textbf{Source}}   & \makecell[l]{An extensive fire damaged the Torch skyscraper in Dubai\\ in February 2015, ruining more than 100 apartments.}    \\
		\multicolumn{1}{l|}{\textbf{Target}}   & \makecell[l]{Rozsáhlý požár poškodil dubajský mrakodrap The Torch \\rovněž v únoru 2015, zničil přes sto bytů. }     \\
		\multicolumn{1}{l|}{\textbf{Proposed}} & \makecell[l]{Rozsáhlý požár poškodil mrakodrap v Dubaji v únoru 2015\\ a zničil více než sto bytů.   }              \\
		\multicolumn{1}{l|}{\textbf{Baseline}} & \makecell[l]{Rozsáhlý požár poškodil průmyslovou výrobu mrakolestných\\ mrakodrapů v únoru, dva tisíce patnáct a zničil více než sto bytů.     } \\
		\textbf{}                              &                                                           \\
		\multicolumn{1}{l|}{\textbf{Source}}   & \makecell[l]{Dehydration manifests itself as a complete loss of\\ self-sufficiency.}     \\
		\multicolumn{1}{l|}{\textbf{Target}}   & \makecell[l]{Dehydratace se projeví jako naprostá ztráta soběstačnosti.} \\
		\multicolumn{1}{l|}{\textbf{Proposed}} & \makecell[l]{Hydratace se projevuje jako naprostá ztráta sebeúčinku.    }  \\
		\multicolumn{1}{l|}{\textbf{Baseline}} & \makecell[l]{Dehydratace se projevuje jako úplná ztráta soběstačnosti.   }
	\end{tabular}
	\caption{Output samples from English to Czech SLT.}
	\label{tab:en_cs_sample}
\end{table}

\section{Conclusion}
\label{slt:conclusion}
We proposed an SLT pipeline with phonemes as intermediate representation. We successfully trained the models for English and Czech (in both directions). Further, we compared these models with baselines (traditional ASR to graphemes followed by an NMT model). In both direction, the proposed SLT pipeline achieves slightly higher SacreBLEU score. 

We also assessed the translation quality subjectively using a random experiment. In this case, the baseline system performed better in the direction from English to Czech, and in the Czech to English direction performed the proposed SLT system better. 

We believe that the proposed system has a potential in the future. Therefore, we would like to review the adaptation of the system to the errors produced by the acoustic model. Also, other techniques known from neural machine translation can be applied (e.g., backtranslation). We are convinced, the proposed systems are capable of even better translation quality. 